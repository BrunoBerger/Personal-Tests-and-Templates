%%%%%%%%%%%%%%%%%%%%%%%%%%%%%
\section{Style}
\pagenumbering{gobble} % remove pagenumbering
\usepackage[doublespacing]{setspace} % singlespacing / onehalfspacing / doublespacing

%%%%%%%%%%%%%%%%%%%%%%%%%%%%%
\section{Figures}

%%%
%Standard figure:
\begin{figure}[H]
  \centering
  \includegraphics[width=0.4\linewidth]{rXUndYStatSem.png}
  \caption{Abweichung visualisiert\cite{statistikSeminar}}
  \label{fig:rUndXY}
\end{figure}

\subsection{Double Column}

% For spanning a single column:
\begin{figure}[tbp]
	\centering\includegraphics[width=1.0\columnwidth]{example.png}
\end{figure}

% For spanning the whole page:
\begin{figure*}[tbp]
	\centering\includegraphics[width=1.0\textwidth]{example.png}
\end{figure*}

%%%
\subsection{Wrapfigure}

\begin{wrapfigure}{l}{0.7\linewidth}
  \includegraphics[width=\linewidth]{./lion-logo.jpg}
  \caption{This is the former Share\LaTeX{} logo}
  \label{fig:_name_}
\end{wrapfigure}

\begin{wrapfigure}{r}{0.5\textwidth}
    \vspace{-25pt}
  \centering{\includegraphics[width=\linewidth]{triangulation dublin.png}}
    \vspace{-20pt}
  \caption{Triangulation \cite{poleTagging}}
  \label{fig:triang}
    \vspace{-6pt}
\end{wrapfigure}

\subsection{Multiple figures}

% Standard
\begin{figure}[!tbp]
  \centering
  \begin{minipage}[b]{0.4\textwidth}
    \includegraphics[width=\textwidth]{flower1.jpg}
    \caption{Flower one.}
  \end{minipage}
  \hfill
  \begin{minipage}[b]{0.4\textwidth}
    \includegraphics[width=\textwidth]{flower2.jpg}
    \caption{Flower two.}
  \end{minipage}
\end{figure}

% With subcaptions (a,b) and same space to each other and to the sides
\begin{figure}[H]
\centering
    \hspace*{\fill}%
    \begin{subfigure}[b]{0.525\textwidth}
            \includegraphics[width=\textwidth]{Figures/FromPW/A_ElderlyStartsCall-ColouredAndCompressed.jpg}
            \caption{An elderly starts a  call}
            \label{fig:A_ElderlyStartsCall}
    \end{subfigure}%
    \hfill
    \begin{subfigure}[b]{0.455\textwidth}
            \centering
            \includegraphics[width=\textwidth]{Figures/FromPW/A_RelativeStartsCall-ColouredAndCompressed.jpg}
            \caption{Relative starts a  call}
            \label{fig:A_RelativeStartsCall}
    \end{subfigure}
    \hspace*{\fill}%
    \caption{Activity diagrams from the project work \cite{projectwork}}
    \label{fig:Activites-from-PW}
\end{figure}

%%%%%%%%%%%%%%%%%%%%%%%%
\section{Footnotes}

\subsection{normal looking link in footnote}

\footnote
    {\href{}
    {}}

\subsection{To reference the same footnote multiple times}

\makeatletter
\newcommand\footnoteref[1]{\protected@xdef\@thefnmark{\ref{#1}}\@footnotemark}
\makeatother
% Use like this:
\footnote{\label{MyLabel}MyFootnote}
\footnoteref{MyLabel}

\subsection{Decouple the ref in the text vs the actual footnote}
% Source: https://tex.stackexchange.com/questions/674175/how-to-make-a-footnote-appear-on-the-left-column

% Place this rougly where the footnote should end up
\footnotetext[\numexpr \value{footnote}+1\relax]{Test footnote B}

% where normally \footnote{} would be:
\footnotemark

%%%%%%%%%%%%%%%%%%%%%%%
\section{Standard-citation}

@misc{REF,
  title = {TITLE},
  author={},
  organization={},
  year={},
  month={01},
  howpublished = {URL},
  note = {Accessed: 2020-04-13}
}

%%%%%%%%%%%%%%%%%%%%%%%%%%%%%
\section{Often used packages}

\usepackage{biblatex}               % citations
\usepackage{dirtytalk}              % quotation marks
\usepackage{float}                  % force image to appear in specific place in text
\usepackage{graphicx}               % images next to each other
\usepackage[hidelinks]{hyperref}    % better hyperlinks
\usepackage{minted}                 % Code linting
\usepackage{multirow}               % cells that span multiple rows in tables
\usepackage{wrapfig}                % embed images into text

\addbibresource{bibfile.bib}
\graphicspath{{Figures/}}

%%%%%%%%%%%%%%%%%%%%%%%%%%%%%
\section{List of footnotes}

\newcommand{\listfootnotesname}{\Large List of Footnotes}% 'List of Footnotes' title
\newlistof{footnotes}{fnt}{\listfootnotesname}% New 'List of...' for footnotes
\let\oldfootnote\footnote % Save the old \footnote{...} command
\renewcommand\footnote[1]{% Redefine the new footnote to also add 'List of Footnote' entries.
    \refstepcounter{footnotes}% Add and step a reference to the footnote/counter.
    \oldfootnote{#1}% Make a regular footnote.
    \addcontentsline{fnt}{footnotes}{\protect
\numberline{\thefootnotes}#1}% Add the 'List of...' entry.
}

%%%%%%%%%%%%%%%%%%%%%%%%%%%%%
\section{Author-Credits}

% put in section-name to indicate author:
\hfill\textnormal{\emph{Name}}

% after paragraph if multiple people worked on one section:
% \vfill % add to place text on bottom
\begin{flushright}
\emph{name}
\end{flushright}

%%%%%%%%%%%%%%%%%%%%%%%%%%%%%
\section{Table of contents}

 % place in text for new page in TOC
\addtocontents{toc}{\protect\newpage}


%%%%%%%%%%%%%%%%%%%%%%%%%%%%%
\section{SubSubSub...-Sections}

https://tex.stackexchange.com/questions/186981/is-there-a-subsubsubsection-command


%%%%%%%%%%%%%%%%%%%%%%%%%%%%%
\section{Subtitle (quickNdirty)}

\title{Deep Learning 2 - Project\\
    \vspace{+14pt}
        \Large Interaktionstechnik und Design \\
        HSHL  Sommersemester 2020}
\author{Bruno Berger, MatrNr: 2170706}
\date{22. 06. 2020}


%%%%%%%%%%%%%%%%%%%%%%%%%%%%%
\section{Unterschrift-Feld}
\vfill
\parbox{5cm}{
\hrule
\strut Ort, Datum}
\hfill\parbox{6cm}
{\hrule \strut Bruno Berger}

%%%%%%%%%%%%%%%%%%%%%%%%%%%%%
\section{Change naming}
% Text at the start of an abstract
\renewcommand{\abstractname}{Vorwort}

% Change bibliography-title
\printbibliography[title = {Literature}]
